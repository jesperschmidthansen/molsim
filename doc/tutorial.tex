\documentclass[11pt]{article}

\title{\textsf{molsim} tutorial}
\author{J.S. Hansen}
\date{Februaray 2022, v. 0.9}
  
\begin{document}

\maketitle

\section{Introduction}

\textsf{molsim} is a GNU Octave/Matlab package for molecular dynamics simulation
library. \textsf{molsim} supports simulations of
\begin{itemize}
\item Standard Lennard-Jones systems (solid, liquids, gasses, etc)
\item Molecular systems with bond, angle, and torsion potentials 
\item Confined flow systems, eg., Couette and Poiseuille flows
\item Charged systems using shifted force and Wolf methods
\item Dissipative particle dynamics systems
\item and more
\end{itemize}
The package also supports a series of run-time sampling functionalities.

\bigskip
\noindent \textsf{molsim} is basically a wrapper for the \textsf{seplib}
library, which is a light-weight flexible molecular dynamics simulation library
written in ISO-C99. The library is CPU-based and offers shared memory
parallisation; this parallisaton is supported by the \textsf{molsim}
package. The algorithms used in \textsf{seplib} is based on the books by Allen
\& Tildesley, Rapaport, Frenkel \& Smith, and R. Sadus, see
Ref. \cite{seplib:books}.

\bigskip
\noindent In this text
\begin{verbatim}
>> 
\end{verbatim}
indicates GNU Octave or Matlab command prompt. This 
\begin{verbatim}
$ 
\end{verbatim}
indicates the shell prompt.

\section{Installation}
\subsection{GNU Octave}
GNU Octave's package manager offers a very easy installation. From

\begin{verbatim}
https://github.com/jesperschmidthansen/molsim/
\end{verbatim}

\noindent download and save the current release
\verb!molsim-<version>.tar.gz! in a directory of your choice. Start GNU
Octave and if needed change directory to the directory where the file is saved.
\begin{verbatim}
>> pkg install molsim-<version>.tar.gz 
\end{verbatim}
Check contact by
\begin{verbatim}
>> molsim('hello')
Hello 
\end{verbatim}
In case this fails, check the path where \textsf{molsim} is install by
\begin{verbatim}
>> pkg list molsim
\end{verbatim}
If the path is not in your GNU Octave search path add this using the
\verb!addpath! command.

\subsection{Matlab}
From
\begin{verbatim}
https://github.com/jesperschmidthansen/seplib/
\end{verbatim}
\noindent download and save the current release \verb!seplib-<version>.tar.gz!
in a directory of your choice. Unpack, configure and build the library
\begin{verbatim}
$ tar zxvf seplib-<version>.tar.gz
$ cd seplib
$ ./configure
$ make
$ cd octave
\end{verbatim}
To build the \textsf{mex}-file enter Matlab
\begin{verbatim}
$ matlab -nodesktop
\end{verbatim}
Then build the 
\begin{verbatim}
>> buildmex
\end{verbatim}
Depending on the system this will build a \textsf{molsim.mex<archtype>}
file. You can copy this file to a directory in your Matlab searce path.

\section{First quick example: The Lennard-Jones liquid}
Listing 1 shows the simplest script simulating a standard Lennard-Jones (LJ)
system in the micro-canonical ensemble where number of particles, volume, and
total energy is conserved.

\bigskip

\noindent \textbf{Listing 1}
\begin{verbatim}
% Specify the LJ paramters
cutoff = 2.5; epsilon = 1.0; sigma = 1.0; aw=1.0;

% Set init. position and velocities 10x10x10 particles in box
% with lengths 12x12x12. Configuration stored in start.xyz
molsim('set', 'lattice', [10 10 10], [12 12 12]);

% Load the configuration file
molsim('load', 'xyz', 'start.xyz');

% Main loop
for n=1:10000

  % Reset everything
  molsim('reset');

  % Calculate force between particles of type A (default type)
  molsim('calcforce', 'lj', 'AA', cutoff, sigma, epsilon, aw);

  % Integrate forward in time
  molsim('integrate', 'leapfrog');
 
end

% Free memory allocated
molsim('clear');
\end{verbatim}
Listing 1 is not very useful as no information is printed or saved. Inside the
main loop you can add the command
\begin{verbatim}
if rem(n,100)==0
  molsim('print');
end
\end{verbatim}
to print current iteration number, potential energy per particle, kinetic energy
per particle, total energy per particle, kinetic temperature, and total momentum
to screen every 100 time step.

More information can be retrieved from the simulation, e.g., get the system
energies and pressure 
\begin{verbatim}
[ekin, epot] = molsim('get', 'energies');
press = molsim('get', 'pressure');
\end{verbatim}
and particle positions and velocities
\begin{verbatim}
x = molsim('get', 'positions');
v = molsim('get', 'velocities');
\end{verbatim}
In general, the \verb!molsim! interface is on the form
\begin{verbatim}
molsim(ACTION, SPECIFIER, ARGUMENTS)
\end{verbatim}
where the action can be \verb!get!, \verb!integrate!, and so on, the specifier
is a specification for the action, and arguments are the arguments for the
specifier. 

\subsection{NVT and NPT simulations}
Often you will not have  

\section{The force field}
\section{Molecular systems}
\section{Sampling}
\section{The two parallisation paradigms}

\end{document}
